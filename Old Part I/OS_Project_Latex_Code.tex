\documentclass{article}
\usepackage[utf8]{inputenc}
\documentclass{article}
\usepackage[LGR]{fontenc}
\usepackage{listings}
\usepackage{multicol}
\usepackage[english,greek]{babel}
\usepackage{tikz}
\usepackage{multicol}
\usetikzlibrary{graphdrawing.trees};
\usepackage{algpseudocode}
\title{ΟΣ}
\author{Ορέστης Α Μακρής}
\date{Δεκέμβριος 2022}
\begin{document}

\section{Ασκηση 3}
Ερώτημα 3: Διαδιεργασιακή Επικοινωνία (15 Μονάδες)
Σε μια φαρμακοβιομηχανία υπάρχει μια γραμμή παραγωγής για τη συσκευασία φαρμάκων σε φιαλίδια, όπου λαμβάνουν χώρα οι παρακάτω εργασίες: 
\\

 \begin{itemize}
 \itemΑνάγνωση προγραμματισμένης παραγόμενης ποσότητας φιαλιδίων με φάρμακο (Ε1)
 \itemΠαραγωγή γυάλινων φιαλιδίων (Ε2)
 \itemΠαραγωγή καπακιών για τα φιαλίδια (Ε3)
 \itemΠαραγωγή ετικετών για τα φιαλίδια και χαρτόκουτα (Ε4)
 \itemΠαραγωγή χαρτόκουτων συσκευασίας (Ε5)
 \itemΑπολύμανση γυάλινων φιαλιδίων και καπακιών (Ε6)
 \itemΕισαγωγή ορού στα φιαλίδια (Ε7)
 \itemΕισαγωγή δραστικής ουσίας στα φιαλίδια (Ε8)
 \itemΤοποθέτηση καπακιών στα φιαλίδια (Ε9)
 \itemΚόλληση ετικετών στα φιαλίδια (Ε10)
 \itemΚόλληση ετικετών στις συσκευασίας χαρτόκουτου (Ε11)
  \itemΣυσκευασία των φιαλιδίων σε χαρτόκουτα (Ε12)
\end{itemize}
\\
Θεωρήστε ότι για τις άνω αναφερόμενες εργασίες, η γραμμή παραγωγής προγραμματίζεται για την απαιτούμενη
παραγόμενη ποσότητα φιαλιδίων με φάρμακο. Ακολουθεί η παραγωγή της ανάλογης ποσότητας, σύμφωνα με τον
προγραμματισμό της παραγόμενης ποσότητας, των γυάλινων φιαλιδίων, των καπακιών, των ετικετών φιαλιδίων και
των χαρτόκουτων συσκευασίας. Απολυμαίνονται τα γυάλινα φιαλίδια και τα καπάκια. Εισάγεται ο ορός στα φιαλίδιακαι στη συνέχεια η δραστική ουσία σε αυτά. Τοποθετούνται τα καπάκια στα φιαλίδια και μετά επικολλάται η ετικέτατους. Επικολλώνται οι ετικέτες στα χαρτόκουτα συσκευασίας και πραγματοποιείται η συσκευασία ανά 12-άδες των φιαλιδίων με φάρμακο σε χαρτόκουτα.
\\

\begin{itemize}
    \item [a)]Να σχεδιάσετε το γράφημα προτεραιότητας (precedence graph) που περιγράφει την εκτέλεση των ανωτέρω εργασιών με βάση τους περιορισμούς που προαναφέρθηκαν. (4 Μονάδες)
    \newpage
     \begin{center}
    \resizebox {.7\textwidth } {!} {
     \begin{tikzpicture}[nodes={draw, circle}, ->]
       \node{E1}
        child { node[name=E2] {E2}}
        child { node {E3} 
         child {node [left=0.4cm][name=E6] {E6}
           child { node {E7}  
             child { node {E8} 
              child { node {E9}
               child { node[right=1.6cm][name=E10] {E10}
                child {node[right=0.6cm][name=E12] {E12}}}}}}}} 
        child { node[name=E4] {E4} }
        child { node {E5} 
         child {node[name=E11] {E11}}
         };
         
       \draw (E2) edge (E6);
        \draw (E4) edge (E10);
         \draw (E11) edge (E12);
         
     \end{tikzpicture}
     }
    
   \end{center}
    \newpage
    \item [b)] Να δώσετε τον παράλληλο ψευδοκώδικα συγχρονισμού των ανωτέρω εργασιών, χωρίς τη χρήση σημαφόρων. (3 Μονάδες)
    \\
    \begin{otherlanguage}{english}
      \begin{lstlisting}
        
    E1
    Cobegin
      Begin
         Cobegin
          Begin
            Cobegin
             E2
             E3
            Coend
           E6
           E7
           E8
           E8
           End
         E4
         Coend
       E10
       End
      Begin
       E5
       E11
      End
    Coend
    E12
     \end{lstlisting}
     \end{otherlanguage}
  \item [g)] Να δώσετε, βασιζόμενοι στο γράφο προτεραιότητας που σχεδιάσατε και χρησιμοποιώντας κατάλληλους σημαφόρους, τον ψευδοκώδικα συγχρονισμού των ανωτέρω εργασιών. (3 Μονάδες)
  \\
     
     \begin{otherlanguage}{english}

        Abstract Pseudo-code
        
        var s1, s2, s3, s4, s5, s6, s7, s8, s9, s10, s11, s12, s13, s14 : semaphores;
        \\
        
        s1 = s2 = s3 = s4 = s5 = s6 = s7 = s8 = s9 = s10 = s11 = s12 = s13 = s14 = 0;
        \\          
        
       cobegin
       \begin{sem}
        \item[] E1:begin up(s1);up(s2);up(s3);up(s4);E1;end;
        \item[] E2:begin down(s1);E2;up(s5)end;
        \item[] E3:begin down(s2);E3;up(s6);end;
        \item[] E4:begin down(s3);E4;up(s11);end;
        \item[] E5:begin down(s4);E5;up(s12);end;
        \item[] E6:begin down(s5);down(s6);E6;up(s7);end;
        \item[] E7:begin down(s7);E7;up(s8);end;
        \item[] E8:begin down(s8);E8;up(s9);end;
        \item[] E9:begin down(s9);E9;up(s10);end;
        \item[] E10:begin down(s10);down(s11);E10;up(s14);end;
        \item[] E11:begin down(s12);E11;up(s13);end;
        \item[] E12:begin down(s13);down(s14);E12;end;
      \end{sem}
      \\end;
      \\\begin{multicols}{2}
\\Process E1
\begin{lstlisting}

 repeat
   repeat 12 times
    print('E1')
   forever
  up(S2);
  up(S3);
  up(S4);
 forever;
\end{lstlisting}

\\Process E2
\begin{lstlisting}
 repeat
  down(S1);
   repeat 12 times
    print('E2')
   forever
  up(S5);
 forever;
\end{lstlisting}


\\Process E3
\begin{lstlisting}
 repeat
  down(S2);
   repeat 12 times
    print('E3')
   forever
  up(S6);
 forever;
\end{lstlisting}

\\Process E4
\begin{lstlisting}
 repeat
  down(S3);
   repeat 12 times
    print('E4')
   forever
  up(S11);
 forever;
\end{lstlisting}

\\Process E5
\begin{lstlisting}
 repeat
  down(S4);
   repeat 12 times
    print('E5')
   forever
  up(S12);
 forever;
\end{lstlisting}


\\Process E6
\begin{lstlisting}
 repeat
  down(S5);
  down(S6);
   repeat 12 times
    print('E6')
   forever
  up(S7);
 forever;
\end{lstlisting}

\\Process E7
\begin{lstlisting}
 repeat
  down(S7);
   repeat 12 times
    print('E7')
   forever
  up(S8);
 forever;
\end{lstlisting}

\\Process E8
\begin{lstlisting}
 repeat
  down(S8);
   repeat 12 times
    print('E8')
   forever
  up(S9);
 forever;
\end{lstlisting}

\\Process E9
\begin{lstlisting}
 repeat
  down(S9);
   repeat 12 times
    print('E9')
   forever
  up(S10);
 forever;
\end{lstlisting}


\\Process E10
\begin{lstlisting}
 repeat
  down(S10);
  down(S11);
   repeat 12 times
    print('E10')
   forever
  up(S14);
 forever;
\end{lstlisting}

\\Process E11
\begin{lstlisting}
 repeat
  down(S12);
   repeat 12 times
    print('E11')
   forever
  up(S13);
 forever;
\end{lstlisting}

\\Process E12
\begin{lstlisting}
 repeat
  down(S13);
  down(s14);
    print('E11')
 forever;
\end{multicols}
\end{lstlisting}
\end{multicols}
\end{otherlanguage}
\\

\item [(δ)]Χρησιμοποιήστε το ελάχιστο πλήθος σημαφόρων και ορθές αρχικοποιήσεις σε αυτούς (5 Μονάδες)

\begin{otherlanguage}{english}

        Abstract Pseudo-code
        
        var s1, s2, s3, s4, s5, s6, s7, s8 : semaphores;
        \\
        
        s1 = s2 = s3 = s4 = s5 = s6 = s7 = s8  = 0;
        \\          
        
       cobegin
       \begin{sem}
        \item[] E1:begin up(s1);E1;end;
        \item[] E2:begin down(s1);E2;up(s2)end;
        \item[] E3:begin down(s1);E3;up(s2);end;
        \item[] E4:begin down(s1);E4;up(s6);end;
        \item[] E5:begin down(s1);E5;up(s7);end;
        \item[] E6:begin down(s2);E6;up(s3);end;
        \item[] E7:begin down(s3);E7;up(s4);end;
        \item[] E8:begin down(s4);E8;up(s5);end;
        \item[] E9:begin down(s5);E9;up(s6);end;
        \item[] E10:begin down(s6);E10;up(s8);end;
        \item[] E11:begin down(s7);E11;up(s8);end;
        \item[] E12:begin down(s8);E12;end;
      \end{sem}
      \\end;
      \\\begin{multicols}{2}
\\Process E1
\begin{lstlisting}
 repeat
   repeat 12 times
    print('E1')
   forever
  up(S1);
 forever;
\end{lstlisting}

\\Process E2
\begin{lstlisting}
 repeat
  down(S1);
   repeat 12 times
    print('E2')
   forever
  up(S2);
 forever;
\end{lstlisting}


\\Process E3
\begin{lstlisting}
 repeat
  down(S1);
   repeat 12 times
    print('E3')
   forever
  up(S2);
 forever;
\end{lstlisting}

\\Process E4
\begin{lstlisting}
 repeat
  down(S1);
   repeat 12 times
    print('E4')
   forever
  up(S6);
 forever;
\end{lstlisting}

\\Process E5
\begin{lstlisting}
 repeat
  down(S1);
   repeat 12 times
    print('E5')
   forever
  up(S7);
 forever;
\end{lstlisting}


\\Process E6
\begin{lstlisting}
 repeat
  down(S2);
   repeat 12 times
    print('E6')
   forever
  up(S3);
 forever;
\end{lstlisting}

\\Process E7
\begin{lstlisting}
 repeat
  down(S3);
   repeat 12 times
    print('E7')
   forever
  up(S4);
 forever;
\end{lstlisting}

\\Process E8
\begin{lstlisting}
 repeat
  down(S4);
   repeat 12 times
    print('E8')
   forever
  up(S5);
 forever;
\end{lstlisting}

\\Process E9
\begin{lstlisting}
 repeat
  down(S5);
   repeat 12 times
    print('E9')
   forever
  up(S6);
 forever;
\end{lstlisting}


\\Process E10
\begin{lstlisting}
 repeat
  down(S6);
   repeat 12 times
    print('E10')
   forever
  up(S8);
 forever;
\end{lstlisting}

\\Process E11
\begin{lstlisting}
 repeat
  down(S7);
   repeat 12 times
    print('E11')
   forever
  up(S8);
 forever;
\end{lstlisting}

\\Process E12
\begin{lstlisting}
 repeat
  down(S8);
    print('E11')
 forever;
\end{multicols}
\end{lstlisting}
\end{multicols}
\end{otherlanguage}


\end{itemize}
\end{document}
